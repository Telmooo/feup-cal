%----------------------------------------------------------------------------------------
%	PACKAGES AND DOCUMENT CONFIGURATIONS
%----------------------------------------------------------------------------------------
\documentclass[article, a4paper, 12pt, oneside]{memoir}

% Margins
\usepackage[top=3cm,left=2cm,right=2cm,bottom=3cm]{geometry}

% Encondings
\usepackage[utf8]{inputenc}

% Language
\usepackage[portuguese]{babel}

% Graphics and images
\usepackage{graphicx}
	\graphicspath{{../images/}}

% Tables
\usepackage{tabularx}

% Paragraph Spacing
\usepackage{parskip}
\usepackage{indentfirst}
\setlength{\parskip}{0.5cm}

% Hyperreferences
\usepackage{hyperref}

% Repeated commands
\usepackage{expl3}
\ExplSyntaxOn
\cs_new_eq:NN \Repeat \prg_replicate:nn
\ExplSyntaxOff

% Header and Footer Things
\usepackage{wallpaper}
\usepackage{fancyhdr}

% Following code to edit the pagestyle
\pagestyle{fancy}
\fancyhf{}
\rhead{Meat Wagons}
\lhead{\leftmark}
\rfoot{Página \thepage}

% Commands
\usepackage{xargs}

%% Linked Email
\newcommand{\email}[1]{
{\texttt{\href{mailto:#1}{#1}} }
}

%----------------------------------------------------------------------------------------
%	DOCUMENT INFORMATION
%----------------------------------------------------------------------------------------
% Title
\title{\Huge \texttt{Meat Wagons - Transporte de Prisioneiros} }
% Authors
\author{
\LARGE \textbf{Turma 2 Grupo 3}\\\\
\begin{tabular}{l r}
	\email{up201806250@fe.up.pt} & Diogo Samuel Gonçalves Fernandes	\\
	\email{up201806490@fe.up.pt} & Hugo Miguel Monteiro Guimarães \\
	\email{up201806554@fe.up.pt} & Telmo Alexandre Espirito Santo Baptista	\\
\end{tabular}
}

%\institute{Faculdade de Engenharia da Universidade do Porto \\ Bases de Dados (BDAD) - Turma 4, grupo 6}

% Date for the report
\date{\today}

% Table of Contents
\addto\captionsportuguese{\renewcommand*\contentsname{Índice}}

%----------------------------------------------------------------------------------------
%	DOCUMENT
%----------------------------------------------------------------------------------------
\begin{document}
%----------------------------------------------------------------------------------------
%	Front Page
%----------------------------------------------------------------------------------------
% Title Author and Date
\maketitle

% More information for front page
\begin{center}
\textbf{Projeto CAL - 2019/20 - MIEIC}
\Repeat{2}{\linebreak}
\begin{tabular}{l r}
	\textbf{Professor das Aulas Práticas}: & Rosaldo José Fernandes Rossetti
\end{tabular}
\Repeat{4}{\linebreak}
% FEUP Logo
\includegraphics[scale=0.4]{FEUP-logo.jpg}

\end{center}

\newpage
% Header Image
\CenterWallPaper{0.1}{FEUP-logo.jpg}
\addtolength{\wpXoffset}{-7.5cm}
\addtolength{\wpYoffset}{13.8cm}

%----------------------------------------------------------------------------------------
%	TABLE OF CONTENTS
%----------------------------------------------------------------------------------------
\tableofcontents*

\newpage
%----------------------------------------------------------------------------------------
%	CHAPTER 1 - Descrição do Problema
%----------------------------------------------------------------------------------------
\chapter[Descrição do Problema][Descrição do Problema]{Descrição do Problema} \label{\thechapter}

Os transportes de prisioneiros entre diversos estabelecimentos como, por exemplo, as prisões, esquadras e tribunais são feitos usando carrinhas que se encontram adaptadas ao serviço. Estes veículos têm a necessidade de serem altamente resistentes uma vez que é necessário garantir que os prisioneiros não conseguem escapar.

Para este projeto, queremos optimizar o percurso dos veículos de forma a recolher e entregar os prisioneiros nos pontos de interesse. De modo a cumprir o pretendido, é possivel dividir nas seguintes fases:


\subsection{Primeira Iteração - Recolha não seletiva de prisioneiros utilizando uma única carrinha}
	Inicialmente considere que só existe uma única camioneta para realizar todos os serviços.
	Com a primeira iteração pretende-se que apenas uma carrinha vá recolher os prisioneiros a uma dada localização, tendo em conta a urgência da situação. As situações que  sejam mais exigentes serão respondidas primeiro pela carrinha.

	É importante de notar que a recolha só pode ser efetuada se exisrem caminhos que liguem todos os pontos de interesse, ou seja, o grafo necessita de ser conexo.

	Algumas vezes, obras nas vias públicas podem fazer com que certas zonas tornem-se inacessíveis, inviabilizando o acesso ao destino de alguns prisioneiros. Avalie a conectividade do grafo, a fim de identificar pontos de recolha e de entrega com pouca acessibilidade.

\subsection{Segunda Iteração - Recolha seletiva de prisioneiros utilizando uma única carrinha}
	Durante a segunda fase, cada prisioneiro irá ser agrupado com outros prisioneiros sempre que seja possivel, de modo a não exceder a capacidade da carrinha.



\subsection{Terceira Iteração - Recolha seletiva de prisioneiros utilizando várias carrinhas}

	Concluindo, nesta ultima fase vai-se ter em consideração o diverso número de carrinhas que a frota possui. Algumas carrinhas vão diferir de outras, tendo cada carrinha uma determinada função. Por exemplo, vão existir carrinhas especificas para transportar prisioneiros até aos aeroportos, linhas de comboio.


\newpage

%----------------------------------------------------------------------------------------
%	CHAPTER 2 - Formalização do Problema
%----------------------------------------------------------------------------------------
\chapter[Formalização do Problema][Formalização do Problema]{Formalização do Problema} \label{\thechapter}

\section{Dados de Entrada}

$C_i$ - sequência de veículos, sendo $C_i(i)$ o seu i-ésimo elemento. Cada veículo é caraterizado por:
\begin{itemize}
	\item $capacity$ - número de prisioneiros que pode transportar
	\item $type$ - tipo de veículo
	\item $loc$ - localização atual do veículo
\end{itemize}

$R_i$ - sequência de pedidos de transporte de prisioneiros, sendo $R_i(i)$ o seu i-ésimo elemento. Cada pedido é caraterizado por:
\begin{itemize}
	\item $pickup$ - local de recolha dos prisioneiros
	\item $dest$ - local de destino dos prisioneiros
	\item $numPris$ - número de prisioneiros a serem transportados
	\item $type$ - tipo de prisioneiros
\end{itemize}

$G_i = (V_i, E_i)$ - grafo dirigido pesado, composto por:
\begin{itemize}
	\item $V$ - vértices, representando pontos da rede viária, com:
		\begin{itemize}
			\item $ID$ - Identificador único do vértice
			\item $D$ - Densidade populacional no vértice
			\item $Adj \subseteq E$ - arestas que saiem do vértice
		\end{itemize}
	\item $E$ - arestas, representando conexão entre dois pontos da rede viária, com:
		\begin{itemize}
			\item $ID$ - Identificador único da aresta
			\item $W_d$ - peso da aresta em relação à distância (representa a distância entre os dois vértices)
			\item $W_t$ - peso da aresta em relação ao tempo (representa o tempo médio que demora a percorrer a distância entre os dois vértices, considerando o tráfego normal naquela conexão da rede viária)
			\item $open$ - se a conexão entre os vértices está aberta, isto é, se a rua estiver cortada por alguma razão então não é possível utilizar esta conexão
		\end{itemize}
\end{itemize}

$S$ - vértice da central

\section{Dados de Saída}

$G_f = (V_f, E_f)$ - grafo dirigido pesado, tendo $V_f$ e $E_f$ os mesmos atributos que $V_i$ e $E_i$, excluindo atributos específicos do algoritmo utilizado

$C_f$ - sequência de veículos com os serviços a realizar, sendo $C_f(i)$ o seu i-ésimo elemento. Cada veículo é caraterizado por:
\begin{itemize}
	\item $S$ - sequência de serviços a realizar, sendo $S(i)$ o seu i-ésimo elemento. Cada serviço é caraterizado por:
	\begin{itemize}
		\item $emptySeats$ - número de lugares vazios
		\item $R_f$ - sequência de pedidos atendidos, sendo $R_f(i)$ o seu i-ésimo elemento. Cada pedido atendido é caraterizado por:
		\begin{itemize}
			\item $pickupHour$ - hora de chegada ao local de recolha
			\item $destHour$ - hora de chegada ao local de destino
		\end{itemize}
		\item $P = { e ~ \epsilon ~ E_i }$ - sequência de arestas a percorrer, sendo $P(i)$ o seu i-ésimo elemento
		\item $dist$ - distância percorrida no serviço
		\item $startHour$ - hora esperada de ínicio do serviço
		\item $endHour$ - hora esperada de termino do serviço
	\end{itemize}
\end{itemize}

\section{Restrições}

\subsection{Sobre os dados de entrada}

\begin{itemize}
	\item $\forall i ~ \epsilon ~ [0, \vert C_i \vert [: capacity(C_i(i)) > 0$, uma vez que não faz sentido os veículos não poderem transportar prisioneiros
	\item $\forall i ~ \epsilon ~ [0, \vert C_i \vert [: loc(C_i(i)$ deve pertencer ao mesmo componente fortemente conexo do grafo $G_i$ que o vértice $S$, dado que o veículo tem de ser capaz de voltar à central. $loc(C_i(i))$ ao chegar a um destino de entrega de prisioneiros
	\item $\forall r ~ \epsilon ~ R_i, dest(r)$ deve pertencer ao mesmo componente fortemente conexo do grafo $G_i$ que o vértice $S$, uma vez que o veículo tem de ser capaz de voltar à central
	\item $\forall r ~ \epsilon ~ R_i, numPris(r) > 0$, uma vez que não faz sentido ter um pedido para transportar zero prisioneiros
	\item $\forall e ~ \epsilon ~ E_i, W_d(e) > 0 \wedge W_t(e) > 0$, uma vez que o peso da aresta representa a distância ou o tempo médio necessário para percorrer a aresta, se esta distância ou tempo forem zero estaremos num ciclo no mesmo vértice
	\item $\forall e ~ \epsilon ~ E_i, e$ deve ser uma rua ao qual os veículos possam utilizar, ruas que os veículos não tenham permissão para entrar não são incluídas no grafo $G_i$
	\item $S ~ \epsilon ~ V_i$, uma vez que a central é um vértice do grafo $G_i$
\end{itemize}

\subsection{Sobre os dados de saída}

\begin{itemize}
	\item $\vert C_f \vert \leq \vert C_i \vert $ - não se pode usar mais veículos que os disponíveis
	\item $\forall v_f ~ \epsilon ~ V_f, \exists v_i ~ \epsilon ~ V_i$ tal que $v_i$ e $v_f$ têm os mesmos valores para todos os atributos, com exceção de atributos especificos aos algoritmos utilizados
	\item $\forall e_f ~ \epsilon ~ E_f, \exists e_i ~ \epsilon ~ E_i$ tal que $e_i$ e $e_f$ têm os mesmo valores para todos os atributos, com exceção de atributos especificos aos algoritmos utilizados
	\item $\forall c ~ \epsilon ~ C_f, \forall s ~ \epsilon ~ S(c), 0 \leq emptySeats < capacity(c)$ pois cada serviço deve ter pelo menos um prisioneiro, e não pode haver sobrelotação do veículo
	\item $\forall c ~ \epsilon ~ C_f, \forall s ~ \epsilon ~ S(c), \vert R_f(s) \vert > 0$ uma vez que só faz sentido realizar um serviço se for houver um pedido de transporte de prisioneiros
	\item $\forall c ~ \epsilon ~ C_f, \forall s ~ \epsilon ~ S(c), endHour(s) > startHour(s)$
	\item $\forall c ~ \epsilon ~ C_f, \forall s ~ \epsilon ~ S(c), startHour(s) < pickupHour(\forall r ~ \epsilon ~ R_f) < endHour(s) ~ \wedge ~ startHour(s) < destHour(\forall r ~ \epsilon ~ R_f) \leq endHour(s)$
\end{itemize}

\section{Função objetivo}

\end{document}
