%----------------------------------------------------------------------------------------
%	PACKAGES AND DOCUMENT CONFIGURATIONS
%----------------------------------------------------------------------------------------
\documentclass[article, a4paper, 12pt, oneside]{memoir}

% Margins
\usepackage[top=3cm,left=2cm,right=2cm,bottom=3cm]{geometry}

% Encondings
\usepackage[utf8]{inputenc}

% Language
\usepackage[portuguese]{babel}

% Graphics and images
\usepackage{graphicx}
	\graphicspath{{../images/}}

% Tables
\usepackage{tabularx}

% Paragraph Spacing
\usepackage{parskip}
\usepackage{indentfirst}
\setlength{\parskip}{0.5cm}

% Hyperreferences
\usepackage{hyperref}

% Repeated commands
\usepackage{expl3}
\ExplSyntaxOn
\cs_new_eq:NN \Repeat \prg_replicate:nn
\ExplSyntaxOff

% Header and Footer Things
\usepackage{wallpaper}
\usepackage{fancyhdr}

% Following code to edit the pagestyle
\pagestyle{fancy}
\fancyhf{}
\rhead{Meat Wagons}
\lhead{\leftmark}
\rfoot{Página \thepage}

% Commands
\usepackage{xargs}

%% Linked Email
\newcommand{\email}[1]{
{\texttt{\href{mailto:#1}{#1}} }
}

%----------------------------------------------------------------------------------------
%	DOCUMENT INFORMATION
%----------------------------------------------------------------------------------------
% Title
\title{\Huge \texttt{Meat Wagons - Transporte de Prisioneiros} }
% Authors
\author{
\LARGE \textbf{Turma 2 Grupo 3}\\\\
\begin{tabular}{l r}
	\email{up201806250@fe.up.pt} & Diogo Samuel Gonçalves Fernandes	\\
	\email{up201806490@fe.up.pt} & Hugo Miguel Monteiro Guimarães \\
	\email{up201806554@fe.up.pt} & Telmo Alexandre Espirito Santo Baptista	\\
\end{tabular}
}

%\institute{Faculdade de Engenharia da Universidade do Porto \\ Bases de Dados (BDAD) - Turma 4, grupo 6}

% Date for the report
\date{\today}

% Table of Contents
\addto\captionsportuguese{\renewcommand*\contentsname{Índice}}

%----------------------------------------------------------------------------------------
%	DOCUMENT
%----------------------------------------------------------------------------------------
\begin{document}
%----------------------------------------------------------------------------------------
%	Front Page
%----------------------------------------------------------------------------------------
% Title Author and Date
\maketitle

% More information for front page
\begin{center}
\textbf{Projeto CAL - 2019/20 - MIEIC}
\Repeat{2}{\linebreak}
\begin{tabular}{l r}
	\textbf{Professor das Aulas Práticas}: & Rosaldo José Fernandes Rossetti
\end{tabular}
\Repeat{4}{\linebreak}
% FEUP Logo
\includegraphics[scale=0.4]{FEUP-logo.jpg}

\end{center}

\newpage
% Header Image
\CenterWallPaper{0.1}{FEUP-logo.jpg}
\addtolength{\wpXoffset}{-7.5cm}
\addtolength{\wpYoffset}{13.8cm}

%----------------------------------------------------------------------------------------
%	TABLE OF CONTENTS
%----------------------------------------------------------------------------------------
\tableofcontents*

\newpage
%----------------------------------------------------------------------------------------
%	CHAPTER 1 - Descrição do problema
%----------------------------------------------------------------------------------------
\chapter[Descrição do Problema][Descrição do Problema]{Descrição do Problema} \label{\thechapter}

Os transportes de prisioneiros entre diversos estabelecimentos como, por exemplo, as prisões, esquadras e tribunais são feitos usando carrinhas que se encontram adaptadas ao serviço. Estes veículos têm a necessidade de serem altamente resistentes uma vez que é necessário garantir que os prisioneiros não conseguem escapar.

Para este projeto, queremos optimizar o percurso dos veículos de forma a recolher e entregar os prisioneiros nos pontos de interesse. De modo a cumprir o pretendido, é possivel dividir nas seguintes fases:


\subsection{Primeira Iteração - Recolha não seletiva de prisioneiros utilizando uma única carrinha}
	Inicialmente considere que só existe uma única camioneta para realizar todos os serviços.
	Com a primeira iteração pretende-se que apenas uma carrinha vá recolher os prisioneiros a uma dada localização, tendo em conta a urgência da situação. As situações que  sejam mais exigentes serão respondidas primeiro pela carrinha. 
	
	É importante de notar que a recolha só pode ser efetuada se exisrem caminhos que liguem todos os pontos de interesse, ou seja, o grafo necessita de ser conexo.
	
	Algumas vezes, obras nas vias públicas podem fazer com que certas zonas tornem-se inacessíveis, inviabilizando o acesso ao destino de alguns prisioneiros. Avalie a conectividade do grafo, a fim de identificar pontos de recolha e de entrega com pouca acessibilidade.

\subsection{Segunda Iteração - Recolha seletiva de prisioneiros utilizando uma única carrinha}
	Durante a segunda fase, cada prisioneiro irá ser agrupado com outros prisioneiros sempre que seja possivel, de modo a não exceder a capacidade da carrinha.
	 


\subsection{Terceira Iteração - Recolha seletiva de prisioneiros utilizando várias carrinhas}

	Concluindo, nesta ultima fase vai-se ter em consideração o diverso número de carrinhas que a frota possui. Algumas carrinhas vão diferir de outras, tendo cada carrinha uma determinada função. Por exemplo, vão existir carrinhas especificas para transportar prisioneiros até aos aeroportos, linhas de comboio.


\newpage

\end{document}
